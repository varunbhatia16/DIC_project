\documentclass[10pt]{article}
\usepackage{../pplmanual}
%%% Commonly Needed packages
\usepackage{graphicx,color,calc}
\usepackage{fancyvrb}
\usepackage{makeidx}
\usepackage{alltt}
\usepackage{html}
\usepackage{hyphenat}
\usepackage{listings}
\usepackage{xspace} %<- creates problems with other hyperlink packages like "html"
\usepackage{hyperref}
\hypersetup{
    colorlinks,%
    citecolor=black,%
    filecolor=black,%
    linkcolor=black,%
    urlcolor=magenta
}


%%% Commands for uniform looks of C++, Charm++, and Projections
\newcommand{\CC}{C\hbox{++}\xspace}
\newcommand{\emCC}{C\hbox{\em++}\xspace}
\newcommand{\charmpp}{Charm++\xspace}
\newcommand{\charm}{Charm++\xspace}
\newcommand{\charmc}{\texttt{charmc}\xspace}
\newcommand{\projections}{\textrm{Projections}\xspace}
\newcommand{\converse}{Converse\xspace}
\newcommand{\ampi}{\textup{AMPI}\xspace}
\newcommand{\tempo}{\textsc{TeMPO}\xspace}
\newcommand{\irecv}{\textsl{iRecv}\xspace}
\newcommand{\sdag}{\textsl{Structured Dagger}\xspace}
\newcommand{\jade}{Jade\xspace}
\newcommand{\ci}{\emph{ci}\xspace}

%%% Commands to produce margin symbols
\newcommand{\new}{\marginpar{\fbox{\bf$\mathcal{NEW}$}}}
\newcommand{\important}{\marginpar{\fbox{\bf\Huge !}}}
\newcommand{\experimental}{\marginpar{\fbox{\bf\Huge $\beta$}}}

%%% Commands for manual elements
\newcommand{\zap}[1]{ }
\newcommand{\function}[1]{{\noindent{\textsf{#1}}\\}}
\newcommand{\cmd}[1]{{\noindent{\textsf{#1}}\\}}
\newcommand{\args}[1]{\hspace*{2em}{\texttt{#1}}\\}
\newcommand{\prototype}[1]{\vspace{0.2in}\index{#1}}
\newcommand{\param}[1]{{\texttt{#1}}}
\newcommand{\kw}[1]{{\nohyphens{\textsf{#1}}\index{#1}}}
\newcommand{\uw}[1]{#1}
\newcommand{\desc}[1]{\indent{#1}}
\newcommand{\note}[1]{(\textbf{Note:} #1)}
\newcommand{\term}[1]{{\bf #1}\index{#1}}

% Explicitly state that part numbering uses uppercase roman numerals
% This is just to keep latex2html from barfing and causing a filename
% collision between part II and chapter 2 etc. This will work as long
% as we don't have an appendix section 9 (which would be alphabetized
% to I (and again conflict with part 1).
\renewcommand{\thepart}{\Roman{part}}

\newcommand{\gitweb}[2]{http://charm.cs.illinois.edu/cgi-bin/gitweb2.cgi?p=charm.git;hb=HEAD;a=#1;f=#2}

\newcommand{\gitwebref}[3]{\href{\gitweb{#1}{#2/charm\%2B\%2B/#3}}{\tt #2/charm++/#3}}
\newcommand{\examplereffile}[1]{\gitwebref{blob}{examples}{#1}}
\newcommand{\examplerefdir}[1]{\gitwebref{tree}{examples}{#1}}
\newcommand{\testreffile}[1]{\gitwebref{blob}{tests}{#1}}
\newcommand{\testrefdir}[1]{\gitwebref{tree}{tests}{#1}}

\begin{htmlonly}
\renewcommand{\examplereffile}[1]{\texttt{\emph{examples/charm++/#1}}}
\renewcommand{\examplerefdir}[1]{\texttt{examples/charm++/#1}}
\renewcommand{\testreffile}[1]{\texttt{\emph{tests/charm++/#1}}}
\renewcommand{\testrefdir}[1]{\texttt{tests/charm++/#1}}
\end{htmlonly}
\makeindex


\makeindex

\title{\charmpp\\ Iterative Finite Element Matrix (IFEM) Library\\ Manual}
\version{1.2}
\credits{
Initial version of \charmpp{} Finite Element Framework was developed
by Orion Lawlor in the spring of 2003.
}

\begin{document}

\maketitle

\section{Introduction}

This manual presents the Iterative Finite Element Matrix (IFEM) library,
a library for easily solving matrix problems derived from 
finite-element formulations.  The library is designed to be matrix-free,
in that the only matrix operation required is matrix-vector product,
and hence the entire matrix need never be assembled

IFEM is built on the mesh and communication capabilities of the Charm++ 
FEM Framework, so for details on the basic runtime, problem setup, and 
partitioning see the FEM Framework manual.


\subsection{Terminology}

A FEM program manipulates elements and nodes. An \term{element} is a portion of
the problem domain, also known as a cell, and is typically some simple shape 
like a triangle, square, or hexagon in 2D; or tetrahedron or rectangular solid in 3D.  
A \term{node} is a point in the domain, and is often the vertex of several elements.  
Together, the elements and nodes form a \term{mesh}, which is the 
central data structure in the FEM framework.  See the FEM manual for details.

% --------------- Solvers -----------------
\section{Solvers}
\label{sec:solver}
A IFEM \term{solver} is a subroutine that controls the search for the solution.

Solvers often take extra parameters, which are listed in a type called 
in C \kw{ILSI\_Param}, which in Fortran is an array of \kw{ILSI\_PARAM} doubles.  
You initialize these solver parameters using the subroutine \kw{ILSI\_Param\_new}, 
which takes the parameters as its only argument.  The input and output
parameters in an \kw{ILSI\_Param} are listed in Table~\ref{table:solverIn}
and Table~\ref{table:solverOut}.

\begin{table}[hh]
\begin{center}
\begin{tabular}{|l|l|l|}\hline
C Field Name & Fortran Field Offset & Use\\\hline
maxResidual & 1 & If nonzero, termination criteria: magnitude of residual. \\
maxIterations & 2 & If nonzero, termination criteria: number of iterations. \\
solverIn[8] & 3-10 & Solver-specific input parameters. \\
\hline
\end{tabular}
\end{center}
\caption{\kw{ILSI\_Param} solver input parameters.}
\label{table:solverIn}
\end{table}

\begin{table}[hh]
\begin{center}
\begin{tabular}{|l|l|l|}\hline
C Field Name & Fortran Field Offset & Use\\\hline
residual & 11 & Magnitude of residual of final solution. \\
iterations & 12 & Number of iterations actually taken. \\
solverOut[8] & 13-20 & Solver-specific output parameters. \\
\hline
\end{tabular}
\end{center}
\caption{\kw{ILSI\_Param} solver output parameters.}
\label{table:solverOut}
\end{table}


\subsection{Conjugate Gradient Solver}

The only solver currently written using IFEM is the conjugate gradient solver.
This linear solver requires the matrix to be real, symmetric and positive definite.

Each iteration of the conjugate gradient solver requires one matrix-vector product
and two global dot products.  For well-conditioned problems, the solver typically 
converges in some small multiple of the diameter of the mesh--the number of elements
along the largest side of the mesh.

You access the conjugate gradient solver via the subroutine name \kw{ILSI\_CG\_Solver}.


% -------------- Shared-node ----------------
\section{Solving Shared-Node Systems}

Many problems encountered in FEM analysis place the entries of the
known and unknown vectors at the nodes--the vertices of the domain.
Elements provide linear relationships between the known and unknown node values, 
and the entire matrix expresses the combination of all these element relations.

For example, in a structural statics problem, we know the net force at
each node, $f$, and seek the displacements of each node, $u$.  Elements 
provide the relationship, often called a stiffness matrix $K$, between 
a nodes' displacments and its net forces:

\[
	f=K u
\]

We normally label the known vector $b$ (in the example, the forces), the unknown
vector $x$ (in the example, the displacments), and the matrix $A$:

\[
	b=A x
\]


IFEM provides two routines for solving problems of this type.  The first routine, \kw{IFEM\_Solve\_shared}, solves for the entire $x$ vector based on the known values of the $b$ vector.  The second, \kw{IFEM\_Solve\_shared\_bc}, allows certain entries in the $x$ vector to be given specific values before the problem is solved, creating values for the $b$ vector.


\newpage
\subsection{IFEM\_Solve\_shared}

\prototype{IFEM\_Solve\_shared}
\function{void IFEM\_Solve\_shared(ILSI\_Solver s,ILSI\_Param *p,
	int fem\_mesh, int fem\_entity,int length,int width,
	IFEM\_Matrix\_product\_c A, void *ptr, const double *b, double *x);}
\function{subroutine IFEM\_Solve\_shared(s,p,
	fem\_mesh,fem\_entity,length,width,
	A,ptr,b,x)}
	\args{external solver subroutine :: s}
	\args{double precision, intent(inout) :: p(ILSI\_PARAM)}
	\args{integer, intent(in) :: fem\_mesh, fem\_entity, length,width}
	\args{external matrix-vector product subroutine :: A}
	\args{TYPE(varies), pointer :: ptr }
        \args{double precision, intent(in) :: b(width,length)}
        \args{double precision, intent(inout) :: x(width,length)}

This routine solves the linear system $A x = b$ for the unknown vector $x$.  \uw{s} and \uw{p} give the particular linear solver to use, and are described in more detail in Section~\ref{sec:solver}.  \uw{fem\_mesh} and \uw{fem\_entity} give the FEM framework mesh (often \kw{FEM\_Mesh\_default\_read()}) and entity (often \kw{FEM\_NODE}) with which the known and unknown vectors are listed.

\uw{width} gives the number of degrees of freedom (entries in the vector) per node. For example, if there is one degree of freedom per node, width is one.  \uw{length} should always equal the number of FEM nodes.

\uw{A} is a local matrix-vector product routine you must write.  Its interface is described in Section~\ref{sec:mvp}. \uw{ptr} is a pointer passed down to \uw{A}--it is not otherwise used by the framework.  

\uw{b} is the known vector.  \uw{x}, on input, is the initial guess for the unknown vector.  On output, \uw{x} is the final value for the unknown vector.  \uw{b} and \uw{x} should both have length * width entries.  In C, DOF $i$ of node $n$ should be indexed as $x[n*$\uw{width}$+i]$.  In Fortran, these arrays should be allocated like \uw{x(width,length)}.

When this routine returns, \uw{x} is the final value for the unknown vector, and the output values of the solver parameters \uw{p} will have been written.

\begin{alltt}
// C++ Example
  int mesh=FEM_Mesh_default_read();
  int nNodes=FEM_Mesh_get_length(mesh,FEM_NODE);
  int width=3; //A 3D problem
  ILSI_Param solverParam;
  struct myProblemData myData;
  
  double *b=new double[nNodes*width];
  double *x=new double[nNodes*width];
  ... prepare solution target b and guess x ...
  
  ILSI_Param_new(&solverParam);
  solverParam.maxResidual=1.0e-4;
  solverParam.maxIterations=500; 
  
  IFEM_Solve_shared(IFEM_CG_Solver,&solverParam,
         mesh,FEM_NODE, nNodes,width,
         myMatrixVectorProduct, &myData, b,x);
  
! F90 Example
  include 'ifemf.h'
  INTEGER :: mesh, nNodes,width
  DOUBLE PRECISION, ALLOCATABLE :: b(:,:), x(:,:)
  DOUBLE PRECISION :: solverParam(ILSI_PARAM)
  TYPE(myProblemData) :: myData
  
  mesh=FEM_Mesh_default_read()
  nNodes=FEM_Mesh_get_length(mesh,FEM_NODE)
  width=3   ! A 3D problem
  
  ALLOCATE(b(width,nNodes), x(width,nNodes))
  ... prepare solution target b and guess x ..
  
  ILSI_Param_new(&solverParam);
  solverParam(1)=1.0e-4;
  solverParam(2)=500; 
  
  IFEM_Solve_shared(IFEM_CG_Solver,solverParam,
         mesh,FEM_NODE, nNodes,width,
         myMatrixVectorProduct, myData, b,x);

\end{alltt}

\newpage
\subsubsection{Matrix-vector product routine}
\label{sec:mvp}

IFEM requires you to write a matrix-vector product routine that will evaluate $A x$ for various vectors $x$.  You may use any subroutine name, but it must take these arguments:

\prototype{IFEM\_Matrix\_product}
\function{void IFEM\_Matrix\_product(void *ptr,int length,int width,
	const double *src, double *dest);}
\function{subroutine IFEM\_Matrix\_product(ptr,length,width,src,dest)}
  \args{TYPE(varies), pointer  :: ptr}
  \args{integer, intent(in)  :: length,width}
  \args{double precision, intent(in) :: src(width,length)}
  \args{double precision, intent(out) :: dest(width,length)}

The framework calls this user-written routine when it requires a matrix-vector product.  This routine should compute $dest = A \, src$, interpreting $src$ and $dest$ as vectors.  \uw{length} gives the number of nodes and \uw{width} gives the number of degrees of freedom per node, as above.

In writing this routine, you are responsible for choosing a representation for the matrix $A$. For many problems, there is no need to represent $A$ explicitly--instead, you simply evaluate $dest$ by looping over local elements, taking into account the values of $src$.  This example shows how to write the matrix-vector product routine for simple 1D linear elastic springs, while solving for displacement given net forces.

After calling this routine, the framework will handle combining the overlapping portions of these vectors across processors to arrive at a consistent global matrix-vector product.

\begin{alltt}
// C++ Example
#include "ifemc.h"

typedef struct \{
  int nElements; //Number of local elements
  int *conn; // Nodes adjacent to each element: 2*nElements entries
  double k; //Uniform spring constant
\} myProblemData;

void myMatrixVectorProduct(void *ptr,int nNodes,int dofPerNode,
          const double *src,double *dest) 
\{
  myProblemData *d=(myProblemData *)ptr;
  int n,e;
  // Zero out output force vector:
  for (n=0;n<nNodes;n++) dest[n]=0;
  // Add in forces from local elements
  for (e=0;e<d->nElements;e++) {
    int n1=d->conn[2*e+0]; // Left node
    int n2=d->conn[2*e+1]; // Right node
    double f=d->k * (src[n2]-src[n1]); //Force
    dest[n1]+=f;
    dest[n2]-=f;
  }
\}


! F90 Example
  TYPE(myProblemData) 
    INTEGER :: nElements
    INTEGER, ALLOCATABLE :: conn(2,:)
    DOUBLE PRECISION :: k
  END TYPE
  
SUBROUTINE myMatrixVectorProduct(d,nNodes,dofPerNode,src,dest)
  include 'ifemf.h'
  TYPE(myProblemData), pointer :: d
  INTEGER :: nNodes,dofPerNode
  DOUBLE PRECISION :: src(dofPerNode,nNodes), dest(dofPerNode,nNodes)
  INTEGER :: e,n1,n2
  DOUBLE PRECISION :: f
  
  dest(:,:)=0.0
  do e=1,d%nElements
    n1=d%conn(1,e)
    n2=d%conn(2,e)
    f=d%k * (src(1,n2)-src(1,n1))
    dest(1,n1)=dest(1,n1)+f
    dest(1,n2)=dest(1,n2)+f
  end do
END SUBROUTINE
\end{alltt}


\newpage
\subsection{IFEM\_Solve\_shared\_bc}

\prototype{IFEM\_Solve\_shared\_bc}
\function{void IFEM\_Solve\_shared\_bc(ILSI\_Solver s,ILSI\_Param *p,
	int fem\_mesh, int fem\_entity,int length,int width,
	int bcCount, const int *bcDOF, const double *bcValue,
	IFEM\_Matrix\_product\_c A, void *ptr, const double *b, double *x);}
\function{subroutine IFEM\_Solve\_shared\_bc(s,p,
	fem\_mesh,fem\_entity,length,width,
	bcCount,bcDOF,bcValue,
	A,ptr,b,x)}
	\args{external solver subroutine :: s}
	\args{double precision, intent(inout) :: p(ILSI\_PARAM)}
	\args{integer, intent(in) :: fem\_mesh, fem\_entity, length,width}
	\args{integer, intent(in) :: bcCount}
	\args{integer, intent(in) :: bcDOF(bcCount)}
	\args{double precision, intent(in) :: bcValue(bcCount)}
	\args{external matrix-vector product subroutine :: A}
	\args{TYPE(varies), pointer :: ptr }
        \args{double precision, intent(in) :: b(width,length)}
        \args{double precision, intent(inout) :: x(width,length)}

Like \kw{IFEM\_Solve\_shared}, this routine solves the linear system $A x = b$ for the unknown vector $x$.  This routine, however, adds support for boundary conditions associated with $x$. These so-called "essential" boundary conditions restrict the values of some unknowns. For example, in structural dynamics, a fixed displacment is such an essential boundary condition.  

The only form of boundary condition currently supported is to impose a fixed value on certain unknowns, listed by their degree of freedom--that is, their entry in the unknown vector.  In general, the $i$'th DOF of node $n$ has DOF number $n*width+i$ in C and $(n-1)*width+i$ in Fortran.  The framework guarantees that, on output, for all $bcCount$ boundary conditions, $x(bcDOF(f))=bcValue(f)$.  

For example, if $width$ is 3 in a 3d problem, we would set node $ny$'s y coordinate to 4.6 and node $nz$'s z coordinate to 7.3 like this:

\begin{alltt}
// C++ Example
  int bcCount=2;
  int bcDOF[bcCount];
  double bcValue[bcCount];
  // Fix node ny's y coordinate
  bcDOF[0]=ny*width+1; // y is coordinate 1
  bcValue[0]=4.6;
  // Fix node nz's z coordinate
  bcDOF[1]=nz*width+2; // z is coordinate 2
  bcValue[1]=2.0;

! F90 Example
// C++ Example
  integer :: bcCount=2;
  integer :: bcDOF(bcCount);
  double precision :: bcValue(bcCount);
  // Fix node ny's y coordinate
  bcDOF(1)=(ny-1)*width+2; // y is coordinate 2
  bcValue(1)=4.6;
  // Fix node nz's z coordinate
  bcDOF(2)=(nz-1)*width+3; // z is coordinate 3
  bcValue(2)=2.0;
\end{alltt}



Mathematically, what is happening is we are splitting the partially unknown vector $x$ into a completely unknown portion $y$ and a known part $f$:
\[ A x = b \]
\[ A (y + f) = b \]
\[ A y = b - A f \]

We can then define a new right hand side vector $c=b-A f$ and solve the new linear system $A y=c$ normally.  Rather than renumbering, we do this by zeroing out the known portion of $x$ to make $y$.  The creation of the new linear system, and the substitution back to solve the original system are all done inside this subroutine.

One important missing feature is the ability to specify general linear constraints on the unknowns, rather than imposing specific values.


\section{Index}
\input{index}
\end{document}
